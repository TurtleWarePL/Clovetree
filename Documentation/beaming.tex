\subsection Beaming

For the purpose of this section, a cluster is a collection of
noteheads that share a stem.  A beam group is a collection of
clusters, the stems of which are connected by beams.

There are two phases to determining correct beaming.  The first phase
consists of deciding, for each beam group, whether it is to be beamed to
any or both of the beam groups to the left and right of it, and if so,
how many beams are to be used.  We refer to this process as beam
grouping.  The next phase takes each of the groups and computes a valid
position and slant for the beams of the group.  We call this process
beams positioning.

\subsubsection Beam Grouping

Beam grouping is best done manually, since automatic beam grouping is
hard and a matter of taste.  If a semi-automatic procedure is desired,
it could be implemented as a macro that computes grouping as a function
of current time signatures. 

For each cluster c, the user indicates two numbers, l(c) and r(c), which
indicate the number of beams to be used to connect c to its left and
right neighbors, respectively.

The number max(l(c), r(c)) determines the duration of the cluster c, so
that zero means means a duration of 1/4 or more depending on the type of
noteheads, one means a duration of 1/8 etc. We write C to mean the
current (or center) cluster, L to mean the left neighbor of C and R to
mean the right neighbor of C.

The number of beams to be drawn between L and C (written bL) and between
C and R (written bR) is determined as follows.  If 

        min(r(L), l(C)) = min(r(C), l(R)) = 0

then we use flags instead of beams for C.  The number of flags to be
drawn is max(l(C), r(C)).  The reason for the use of flags is justified
by the fact that at the same time either L does not want to have a right
beam or C does not want to have a left beam AND either C does not want a
right beam or R does not want a left beam.  Using beams would violate
the preferences of one or more of L, C, and R. 

If instead 

        min(r(L), l(C)) > 0, or
        min(r(C), l(R)) > 0

then we use beams.  The number of full-size beams (i.e. stretching all
the way from one cluster to the other) between L and C is determined as
min(r(L), l(C)), and the number of full-size beams between C and R is
determined as min(r(c), l(R)).  Notice that this calculation would give
the same result if done from the point of view of L and R respectively,
which is a reasonable property. 

We now consider C to have 

        max(min(r(L), l(C)), min(r(C), l(R)))

full-size beams, which is less than or equal to the number that we
want to show, which is

        max(l(C), r(C))

so we have to make up for the difference by drawing 

        max(l(C), r(C)) - max(min(r(L), l(C)), min(r(C), l(R)))

fractional beams.  These beams are always drawn on the side having the
most full-size beams, whether l(C) and r(C) indicate it or not.  Thus if

        min(r(L), l(C)) >  min(r(C), l(R))

then the fractional beams are drawn on the left side of the stem of C,
otherwise, they are drawn on the right side of the stem. 

These rules might seem complicated, but it comes out quite nice at the
end.  For instance, a user that does not want any beams at all simply
always gives l(c) = 0 for all clusters c. That way the expression 

        min(r(L), l(C))

is always zero.  For the first cluster C of a group, typically l(C) = 0,
and r(C) = n, where n is the number of beams that gives C its correct
duration.  For the last cluster of a group, typically l(C) = n,  and r(C)
= 0.  For the interior clusters of a group, typically l(C) = r(C) = n.
Sometimes, however, in a group of (say) four clusters, C1, C2, C3, and C4,
you want all of them to be connected by one beam, but in addition you
want C1 and C2 to be connected by a secondary beams, as well as C3 and
C4. Then you would have 

                l(C1) = r(C4) = 0, 
                r(C1) = l(C2) = 2,
                r(C2) = l(C3) = 1,
                r(C3) = l(C4) = 2.

The method described in this section is simple yet flexible.  Automating
grouping is very hard and manual settings must be allowed.  Instead we
use a method that is essentially manual.  We suggest optional minor
modes that compute grouping semi-automatically, or incrementally during
data entry.  

\subsubsection Beam Positioning

The input to this phase is a sequence of clusters with the number of beams
indicated for each pair of adjacent clusters.  This number is always at
least one, since otherwise we would have more than one group.

A valid beaming of a beam group consists of the slant and position of
each beam of the group.  Since beams are parallel and since their
distances are fixed, the valid beaming is completely determined by the
direction and length of the stems of the two extreme clusters of the
group.  

A beaming B is valid for a beam group G if and only if a B is valid for
the group you get by reflecting G in a vertical line.  A beaming B is
valid for beam group G if and only if B is valid for the group you get
by reflecting G in the middle staff line.  Thus, if all the stems go the
same direction, we only have to consider beaming of groups where all
stems go (say) up and with an upward or horizontal beam slant.

If there is more than one note in a cluster, then the note that is closest
to the beams determine beam direction and slant.  Thus, we only have to
consider beam groups where each cluster has only one note. 

The thickness of a beam is 1/2 of the staff line distance. 

When we measure stem lengths, we count in multiples of staff line
distances from the center of the notehead to the end of the stem.  These
multiples are always an integral number of quarters of staff line
distances.  The distance between a beam and a notehead is counted from
the center of the notehead to the edge of the beam that is furthest away
from the notehead.  Stems are conceptually considered as traversing the
beam.  The distance between a notehead and the closest beam is never
shorter than 2 1/2 times the staff line distance.  Since this distance
includes half of the notehead and the entire beam (both of which are 1/2
time a staff line distance), this means that the space between the
notehead and the beam can be as little as 1 1/2 space, but no less. 

Beams never slant more than about 18 degrees.  It is considered
unprofessional to slant more than that. 

The end of a beam either sits on a staff line (or a ledger line), hangs
under it, or straddles it.  This means that stems come in lengths that
are multiples of 1/4 of a staff line distance.  The difference between
the two endpoints of a beam never exceeds 2 times a staff line distance,
and it is rare that it goes beyond 1 1/4 staff line distance. 

The beam that is furthest from the noteheads (again assuming that all
stems go the same way) is called the primary beam.  Beams that are not
primary are called secondary. 

Assuming we limit ourselves to stems going up, primary beam can not be
positioned too far down.  At most, each of the endpoints hangs the
middle staff line.  A secondary beam can not be positioned too far down
either.  At most, each of the endpoints straddles (double beams) or sits
on (tripple beams and higher) the second line from the bottom.  So even
if all notes are below some considerable number of ledger lines, the
beams never reach below the second staff line from the bottom. 

When some secondary beams are partial, i.e. when they do not stretch
from the leftmost cluster to the rightmost cluster in the group, we have a
big problem.  Since the horizontal distances between the clusters are
determined by the spacing algorithm before beaming is considered, the
difference in vertical position between two ends of such a partial
secondary beam may not be an integral multiple of 1/4 times the staff
line distance.  There is therefore a considerable risk that such a
secondary partial beam will end not on a staff line, but in a space.
When the secondary beam is outside the staff, this is not a problem, but
when it is inside, it is a problem.  The solution is to restrict the
slant of such groups to be less than or equal to 1/2 times the staff
line distance.  In that case, every beam will be permanently attached to
one single staff line, i.e. at every horizontal location, such a beam is
traversed by a staff line, and it is one and the same staff line on the
entire beam.  So if in this case, we compute beam positions as if all
beams were full-size, and then simply draw only the parts of the partial
secondary beams that we were told, then those partial secondary beams
will also, at every horizontal location, be traversed by a staff
line, and it will be the same staff line for the entire partial beam. 

With fractional beams we can be somewhat more liberal.  Since beam slant
is restricted to about 18 degrees, we can be sure that an upward
fractional beam that either straddles or hangs a staff line will still
be attached to that same staff line by the time it ends.

A slant of 3/4 is never used for a beam inside of a staff.  Using such a
space would create an instance of "the wedge problem" [Ross].  For the
same reason, a slant of 1/2 is only used (for upward beams again) if the
left side hangs and the right side sits on the staff line.

It is always possible to find a valid beaming.  In the worst case, we
can always make a horizontal beam with sufficiently long stems.  


For each beam slant (assuming horizontal or upward beams), here are the
possible constellations:

      Slant         Left side       Right side
      -----         ---------       ----------
        
        0               St              St
        0               S               S
        0               H               H
        1/4             St              S
        1/4             H               St
        1/2             H               S
        1               St              St
        1               S               S
        1               H               H
        1 1/4           St              S
        1 1/4           H               St
        1 1/2           H               S
        1 1/2           S               H
        1 3/4           S               St
        1 3/4           St              H
        2               St              St
        2               S               S
        2               H               H

For double beams, there is an additional restriction.  The distance
between the upper edge of the two beams is 3/4 times the staff line
distance.  If again we assume stems up and upward slant of the beams,
then either the primary beam straddles the staff line in which case the
secondary beam sits on the next one down, or the primary beam hangs on
the staff line and the secondary beam straddles the next one. The list
becomes (secondary beam in parenthesis)

      Slant         Left side        Right side
      -----         ---------       ----------
        
        0               St (S)          St (S)
        0               H  (St)         H  (St)
        1/4             H  (St)         St (S)
        1               St (S)          St (S)
        1               H  (St)         H  (St)
        1 1/4           H  (St)         St (S)
        1 3/4           St (S)          H  (St)
        2               St (S)          St (S)
        2               H  (St)         H  (St)

For triple beams, the additional restriction is that the primary beam
must hang, the first secondary must straddle, and the second secondary
must sit, so we get 

      Slant         Left side        Right side
      -----         ---------       ----------
        
        0               H  (St) (S)     H  (St) (S)
        1               H  (St) (S)     H  (St) (S)
        2               H  (St) (S)     H  (St) (S)


For quadruple (and beyond) beams, the distance between two adjacent
beams is stretched from 3/4 so that for n beams 

        (n - 1) * d = n - 3/2

where d is the distance between two adjacent beams.  Notice that with n
= 3, we get d = 3/4 which is the normal distance.  For n = 4 we get d =
5/6, for n = 5, we get d = 7/8 etc.  Again only integral slants can be
used.

Again, if the beams are outside the staff, then we can be more liberal. 
No stretching is necessary and we can use slants other than integral
ones. 


Since most of the restrictions apply only when the beams are inside the
staff, it may seem like we still have an infinite number of
possibilities, but such is not the case.  If the beams are entirely
outside the staff (with stems necessarily pointing towards the staff in
this case), then if all stems are longer than 3 1/2 times the staff line
distance, then we can always shorten all stems until the innermost beam
touches the staff, or until at least one stem is 3 1/2 time the staff
line distance or shorter.  The new solution is better than the original
one. 

So far, we have considered only the case when all stems point the same
direction.  Also, we have assumed the automatic procedure always comes
up with the optimal solution.  Unfortunately, it is very hard to find an
optimal solution in all cases.  We therefore suggest that manual
intervention be possible.  For a cluster, the user can choose whether the
stem should be up, down, or auto, where auto means that the user accepts
the choice of the automatic algorithm.  With this possibility, we can
leave all strange cases to the user.  The algorithm can assume that all
stems should go the same way unless specifically indicated.  Fine tuning
of the beam slant can be accomplished by allowing the user to specify
stem height, or leaving it auto.  Specific stem heights will be
respected only for the extreme clusters.
