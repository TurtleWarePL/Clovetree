%======================================================
\chapter{The beaming algorithm}

The beaming algorithm of \sysname{} was inspired by the book by Ross.
Unfortunately, his method is not easily converted into an algorithm,
and his examples occasionally break his own rules.  We thus had to
invent our own algorithms that is as close to what we believe Ross
intended as we could possibly achieve. 

Generally speaking (we are simplifying the discussion to a single
beam) the beaming problem takes as input the x- and y-positions of a
number of clusters and computes the position of the endpoints of the
beam to connect the clusters. 

\section{Beaming is symmetric}

From studying the examples of Ross, one concludes that beaming
is symmetric, i.e., the solution to a beaming problem is also a
solution to the beaming problem obtained when the beam group is
reflected either in the middle staff line, or in an arbitrary vertical
line.  This makes it necessary to handle only beaming problems with
stems in one direction (say up) and with the first note (say) lower
than the last of the beam group. 

\section{Only the notes closest to the beam count}

Each cluster can be reduced to one with only a single notehead, namely
the one closest to the beam.  In our case, it would be the note on the
highest staff position in each cluster.

\section{Notes determining slope and position}

The slope of the beam is determined only by the outermost clusters
(though Ross shows some examples of exceptions to this rule).  The
position of the beam is translated vertically by an even number of
staff steps, should any of the clusters in the group otherwise have
too short a stem. 

With these observations, we now only have to find the slope and
initial position of a beam of a beam group consisting of two clusters,
each with a single note and where the note of the leftmost cluster is
on the same or a lower staff position than the one of the rightmost
cluster.  

\section{General algorithm for single beam}

In the description that follows, we use the \sysname{} method of naming
staff positions which is that we count in half staff steps above the lowest
line of the staff.  For example, the first ledger line below the staff has
position -2 and the upper line of the staff has position 8. 

The primary beam can not be lower than the middle staff line (position
4), but it can hang under it. 

Here is the rule we use (and which coincides with that of Ross in most
cases):

\begin{itemize}
\item if the rightmost note is strictly lower than then the first
  ledger line below the staff (it has a position of at most -3), then
  the left side always hangs under the middle staff line and the right
  side either hangs under, straddles or sits on that same line.  We vary the
  beam slope according to the position of the left note
  \begin{itemize} 
  \item if it has the same position as the right one, then the right
    side of the beam also hangs under the line
  \item if it is one lower than that of the right note, then the right
    side of the beam straddles the line so that its slope reflects the
    difference in position
  \item otherwise (i.e., the left note is more than one position
    below the right one), then the right side of the beam sits on the
    line. 
  \end{itemize}
\item if instead the rightmost note is ON the first ledger line
  (position -2) we raise the beam a little, but it is still entirely
  on the middle staff line:
  \begin{itemize}
  \item if the left note also has position -2, both sides of the
    beam will straddle the middle line.
  \item if the left note has position -3, then the beam hangs the line
    on the left and straddles the line on the right.
  \item otherwise (the left note is at least two positions lower than
    the right one) the beam hangs on the left side and sits on the
    right side. 
  \end{itemize}
\item if instead the rightmost note is below the lowest staff line
  (position -1) the beam is entirely ether on the middle staff line
  (position 4) or on the one above it (position 6). 
  \begin{itemize}
  \item if the left note has the same position (-1), then the beam
    hangs the line at position 6 at both ends. 
  \item if the left note is one position lower (-2) the beam is on the
    middle staff line (position 4), with the left side straddling it and
    the right side sitting on it. 
  \item otherwise (the left note is at least two positions lower than
    the right one) the beam is again on the middle staff line with the
    left side hanging under it and the right side sitting on it. 
  \end{itemize}
\item if instead the rightmost note is on one of the normal staff
  lines (even position from 0 to 8 inclusive) then the right side of
  the stem always straddles a line (possibly not physically shown) six
  positions higher, which makes the right stem 3 1/4 staff steps
  (6 1/2 positions) long. 
  \begin{itemize}
  \item if the left note has the same position as the right one, the
    left side of the beam also straddles the same line.
  \item if the left note is one position below the right one, the left
    side of the beam hangs under the same line.
  \item if the left note is two positions below the right one, the
    left side of the beam straddles the staff line below the one that
    the right side of the beam straddles.
  \item otherwise (i.e., the left note is at least three positions
    lower than the right note, the left side of the beam hangs under
    the staff line below the one that the right side of the beam
    straddles. 
  \end{itemize}
\item if instead the rightmost note is between two ordinary staff
  lines (odd positions from 1 to 7 inclusive), then the right side of
  the beam always sits on a staff line 5 positions above the right
  note, which makes the right stem 3 staff steps long (6 positions).
  \begin{itemize}
  \item if the left note has the same position as the right one, the
    left side of the beam also sits on the same line
  \item if the left note is one positions lower than the right one, the
    left side of the beam straddles the same line.
  \item if the left note is two positions lower than the right one,
    the left side of the beam hangs under the same line.
  \item otherwise (the left note is at least three positions lower
    than the right one), the left side of the beam straddles the staff
    line immediately below the one of the right side of the beam. 
  \end{itemize}
\item if instead the rightmost note is above the entire staff on an
  even position (on a fictive ledger line above the staff), then the
  right side of the beam sits on a fictive line four positions above
  the note.  That makes the right stem extremely short (2 1/2 staff
  step, or 5 positions)
  \begin{itemize}
  \item if the left note has the same position as the right one, the
    left part of the beam also sits on the same line.
  \item if the left note has a position one below that of the right
    note, the left part of the beam straddles the same line.
  \item if the left note has a position two below that of the right
    note, the left part of the beam hangs under the same line.
  \item if the left note has a position three, four or five below that
    of the right note, the left part of the beam straddles the line
    below the one on the right. 
  \item otherwise (the left note has a position of at least six below
    the right one), the left part of the beam hangs under the staff
    line below the one on the right.
  \end{itemize}
\item otherwise, i.e., the rightmost note is above the entire staff in
  an odd position (below a fictive ledger line above the staff) then
  the right side of the beam always straddles a (fictive) line five
  positions above the note.  This makes the right stem 2 3/4 staff
  step (5 1/2 positions) long.
  \begin{itemize}
  \item if the left note has the same position as the right one, then
    the left part of the beam also straddles the same line
  \item if the left note has a position one lower than the right note,
    then the left side of the beam hangs under the same line
  \item if the left note has a position two lower than the right note, 
    the left side of the beam straddles a line one below the one on
    the right.
  \item if the left note has a position three, four, five, or six
    lower than the note on the right, then the left side of the beam
    hangs under the line below the one on the right.
  \item otherwise (the left note has a position at least seven lower
    than the note on the right), then the left part of the note
    straddles a line two staff steps (four positions) below the line
    on the right.  This is the only case with such a high beam slant. 
  \end{itemize}
\end{itemize}

\section{Algorithms for more than one beam}

Double beaming adds more constraints, so it is easier.

Triple beaming adds even more. 

With more than three beams, things get messy because we cannot
preserve normal spacing between beams. 

\section{Avoiding too high a beam slant}

When the horizontal distance between the two outer notes is too small,
normal beaming cannot be used, since it would result in beam slants
that are esthetically too great.  Ross only gives examples, but from
those examples, one may conclude that beam slants more than 18 degrees
are not used.  Ross also does not give any hints as to how to go about
finding a correct solution in this case, he simply hints that (for
stems up, upward beam) the left part of the beam must be raised a
bit.  Our solution is to recursively try a new beaming with the left
note one position higher up, until a solution is found.  A solution is
always found since ultimately, when the who notes have the same
position, a horizontal beam will always be the result, and a
horizontal beam has a slant of zero degrees. 


