\chapter{Development history}

The idea of \sysname{} can be traced back to 1994, when {\rs} proposed a
DESS project \footnote{A DESS is an obsolete degree (obtained after
five years of study at the university) of the French educational
system which has since been replaced by a \emph{Master} degree}.
This project was conducted by {\kerhoas} (and ...), and resulted
in the first real prototype of \sysname{}.  The main features of this
prototype was that it was written in {\tcl}, and that its \emph{user
  model} (see chapter \ref{user-model}) was quite different from the
current one. 

As a summer internship, what-is his-name continued working on the
project.  This work was important in that it revealed some severe
problems with the user model.

Although {\kerhoas} had determined that {\tcl} was a good choice (I
don't think he would say that today), I (i.e., {\rs}) was unsatisfied
with this solution.

{\rms} had already (correctly, in my opinion) determined that (so
called) scripting languages posed a problem because although they were
supposed to be used as \emph{glued} between modules written in a more
efficient language such as C, and that their implementation was
usually to slow for implementing basic functionality, users preferred
them to the lower-level language.  This created two major problems:
first, applications written in the scripting language were slow
(because of the poor implementation of the scripting language), and
second, the part of the applications written using the scripting
language was unmaintainable (due to the poor abstraction capabilities
of the scripting language).  For that reason, he used a
Lisp-derivative (Emacs Lisp) to create Emacs.  

Emacs Lisp was created at a time when there were many, substantially
different Lisp dialects.  The choice was not obvious to {\rms}, so he
created his own dialect, which was a reasonable solution at the time. 

In 1994, there were two main dialects of Lisp remaining, {\commonlisp} and
{\scheme}.  At the time, I thought {\scheme} was more modern and that
{\commonlisp} was going to lose, so I opted for {\scheme}.

The problem was that {\scheme} did not have any standard library for
graphic user interfaces (determined essential for \sysname{}); in fact it
did not have any standard library for anything higher level at all.
It was thus necessary to choose not only a language, but an
implementation of a language.  Based on my knowledge at the time, I
choose {\elk}.  It had a library for interfacing to {\xwin} which was
determined essential to \sysname{}.

As it turned out, {\elk} had exactly the problems {\rms} had
predicted: it was too slow to use for the main parts of the system, so
one ended up writing major parts in C, which was too low level for
this kind of software. 

At about this time, the {\gnu} project started promoting {\guile}, 
a new {\scheme} dialect to be used in {\gnu} projects.  The main idea
of {\guile} was that, since it was a dialect of {\scheme}, and that
{\scheme} is strictly more powerful that most other languages (mainly
due to its first-class continuations), it would be possible to
translate other languages (such as {\tcl}) in a fairly
straight-forward way to {\guile}, and thus allow for any less powerful
scripting language to be used as extension language for any future
{\gnu} application. 

I jumped on the {\guile} bandwagon and converted \sysname{} to Guile in
around 1995 (with the help of some students, as usual of course).
Unfortunately, the {\guile} project made some very unfortunate
choices, such as basing it on Aubrey Jaffer's {\scm}.  Although {\scm}
was one of the fastest known interpreters for {\scheme} it was still
an interpreter.  And after the {\guile} team had added essentials
(like classes and such) to {\scm} to obtain {\guile}, it was no longer
very fast.  In addition, {\guile} turned out to be a moving target for
years to come, and often obliged me to rewrite parts of \sysname{} that I
thought were done.  I did spend some effort on this, though, writing a
library for accessing {\xwin} and some other important
infrastructure.  Ultimately, {\guile} turned out to be too hard to
track, and it was not clear that it was going to be fast enough to
write an application in, nor that it would ever be stable enough for
me. 

For a while I was stuck and lost interest.  After spending a
sabbatical year in Austin (working with the RScheme research group), I
realized that I had been distracted from the obvious choice of
language many years before.  I started looking into {\commonlisp} for the
first time after having been mainly into {\scheme} since around 1985.
It turned out that {\commonlisp} was the perfect choice for several reasons:
it had a standard (ANSI); it had good, fast implementations that were
free (in the {\gnu} sense of the term); it had a semi-standard
interface to {\xwin}; it did not lack any fundamental features such as
classes and exceptions. 

So, I re-implemented \sysname{} in {\commonlisp} starting late 1998.  I then no
longer had to worry about tracking a ``standard'' or whether my
language implementation was going to be fast enough.  And, I could
safely use techniques from object-oriented programming without
worrying whether my language implementation would support them. 

But even this solution had a major problem.  The interface to {\xwin},
although very good, was not high level enough.  I needed a real
library for graphic user interfaces.  Many such libraries existed,
though not for {\commonlisp} at least not freely available.  In fact {\commonlisp} had
(and still does have) a semi-standard library called {\clim} (Common
Lisp Interface Manager) which turns out to be an order of magnitude
more advanced than such libraries for other languages, and which has a
public specification.  The only (major) problem was that it had no
freely available implementation.  {\commonlisp} vendors had versions that they
charged an arm and a leg for, and which would be unacceptable to \sysname{}
anyway, since not freely available.  

I therefore decided to put the \sysname{} project on hold in order to write
a freely available implementation of {\clim}.  To make a long story
short \footnote{This story is told in great detail in the McCLIM
  documentation, part of the McCLIM software distribution}, it took a
little more than three years, and lots of help from other
contributors, to get the freely available implementation of {\clim} to a
state where it would be good enough to implement Gsharp. 

Starting in 2002, I gradually integrated \sysname{} with {\clim}.  Arnaud
Rouaned worked on it for a while, but it was not until August of 2003
that I finally had time to spend time on it myself again.  During the
first two weeks of August, I re-implemented anti-aliased fonts, an
improved version of the Obseq library, and a new system for graphics
output based on {\clim} recording streams. 
